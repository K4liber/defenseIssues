Jeśli układ posiada pewną symetrię, oznacza to, że równania opisujące ten układ nie zmieniają swojej postaci po dokonaniu przekształceń symetrii.

\underline{Dyskretne przekształcenia symetrii} to takie, których nie można sparametryzować np.
\begin{enumerate}[-]
	
\item Teoria grup i symetrii translacyjnej dla sieci periodycznej w kryształach.
\item Symetria permutacyjna funkcji falowej dla układu wielu ciał - związana z nierozróżnialnością cząstek elementarnych (zamiana miejscami cząstek układu nie zmiłaby równań opisujących układ).
\item Symetria zwierciadlana \textbf{P} związana z przekształceniem odbicia przestrzennego (zmiana znaków składowych przestrzennych wektorów na przeciwne).
\item Odwracalność w czasie \textbf{T} (zmiana znaku czasu w równaniach).
\item Parzystość ładunkowa \textbf{C} (zmiana znaku ładunku).

\end{enumerate}

Elektromagnetyzm, grawitacja i oddziaływania silne są niezmiennicze względem każdej z ostatnich trzech wymienionych symetrii (CPT) osobno, jednakże w przypadku oddziaływań słabych niezmienniczość jest zachowana tylko w przypadku łącznego ich działania CPT (rozpad $ \beta $ łamie symetrie P i C, ale zachowuje połączoną symetrie CP, która dla odmiany jest łamana w przypadku rozpadu mezonów K).

\underline{Symetrie związane z ciągłymi przekształceniami} są bezpośrednio związane z istnieniem zasad zachowania - związek ten opisuje twierdzenie Noether. Zgodnie z tym twierdzeniem, z daną symetrią układu jest związanych tyle praw zachowania, ile ciągłych rzeczywistych parametrów potrzebnych jest do sparametryzowania odpowiadających tej symetrii przekształceń np.

\begin{enumerate}[-]
\item Zasada zachowania energii wynika z symetrii związanej z przesunięciem w czasie - niezmienniczości działania S opisującego ruch danego układu od czasu (t - parametr). Jeżeli układ absorbuje lub emituje energie, wówcząs to działanie jest funkcją czasu (t) - odpowiada to w konsekwencji zmianie energii układu.

\item Zasada zachowania pędu wynika z symetrii związanej z przesunięciem układu w przestrzeni.

\item Zasada zachowania momentu pędu wynika z z symetrii związanej z obrotem układu.

\item Zasada zachowania ładunku wyniki z niezmienniczości funkcji falowej elektronu względem transformacji cechowania.
\end{enumerate}