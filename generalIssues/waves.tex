Falą elektromagnetyczną nazywamy rozchodzące się w przestrzeni zaburzenie pola elektromagnetycznego. Składowa elektryczna i magnetyczna fali indukują się wzajemnie – zmieniające się pole elektryczne wytwarza zmieniające się pole magnetyczne, a z kolei zmieniające się pole magnetyczne wytwarza zmienne pole elektryczne.

Promieniowanie elektromagnetyczne przejawia właściwości falowe ulegając interferencji, dyfrakcji, spełnia prawo odbicia i załamania. W wyniku superpozycji fal elektromagnetycznych może powstać fala stojąca. 

Strumień energii przenoszonej przez falę elektromagnetyczną w każdym punkcie przestrzeni określa wektor Poyntinga zdefiniowany jako:\newline
$ \vec{S} = \frac{1}{\mu_0} \vec{E} \times \vec{B} $,\newline
gdzie:\newline
$ \mu_0 $ - przenikalność magnetyczna próżni,\newline
$ \vec{E} $ - natężenie pola elektrycznego,\newline
$ \vec{B} $ - indukcja pola magnetycznego.

Choć w elektrodynamice klasycznej energię promieniowania elektromagnetycznego uważa się za wielkość ciągłą, zależną jedynie od natężenia pola elektrycznego i indukcji pola magnetycznego, to zjawiska zachodzące na poziomie atomowym dowodzą, że jest ona skwantowana. Energia pojedynczego kwantu jest zależna tylko od częstotliwości fali $ \nu $ i wynosi:\newline
$ E = h \nu $, gdzie \textit{h} - stała Plancka.

Właściwości fal elektromagnetycznych zależą od długości fali. Promieniowaniem elektromagnetycznym o różnej długości fali są:
 
\begin{enumerate}[-]
	\item \underline{Fale radiowe} (długość fali powyżej 1 m) - znajdują bardzo szerokie zastosowanie w telekomunikacji, radiofonii, telewizji, radioastronomii i wielu innych dziedzinach nauki i techniki. Naturalne źródła fal radiowych to między innymi wyładowania atmosferyczne, zorze polarne, radiogalaktyki.
	\item \underline{Mikrofale} (od 1 mm do 1 m) - podstawowe zastosowania mikrofal to łączność (na przykład telefonia komórkowa, radiolinie, bezprzewodowe sieci komputerowe) oraz technika radarowa. Fale zakresu mikrofalowego są również wykorzystywane w radioastronomii, a odkrycie mikrofalowego promieniowania tła miało ważne znaczenie dla rozwoju i weryfikacji modeli kosmologicznych. Wiele dielektryków mocno absorbuje mikrofale, co powoduje ich rozgrzewanie i jest wykorzystywane w kuchenkach mikrofalowych, przemysłowych urządzeniach grzejnych i w medycynie.
	\item \underline{Podczerwień} (od 700 nm do 1 mm) - promieniowanie podczerwone jest nazywane również cieplnym, szczególnie gdy jego źródłem są nagrzane ciała. Każde ciało o temperaturze większej od zera bezwzględnego emituje takie promieniowanie, a ciała o temperaturze pokojowej najwięcej promieniowania emitują w zakresie długości fali rzędu 10 $ \mu m $. Przedmioty o wyższej temperaturze emitują promieniowanie o większym natężeniu i mniejszej długości, co pozwala na zdalny pomiar ich temperatury i obserwację za pomocą urządzeń rejestrujących wysyłane promieniowanie (termowizja).
	\item \underline{Światło widzialne} (od 380 nm do 700 nm) - światło (promieniowanie widzialne) to ta część widma promieniowania elektromagnetycznego, na którą reaguje zmysł wzroku człowieka. Różne zwierzęta mogą widzieć w nieco różnych zakresach. Światło ma bardzo duże znaczenie w nauce i wiele zastosowań w technice. Dziedziny nauki i techniki zajmujące się światłem noszą nazwę optyki.
	\item \underline{Ultrafiolet} (od 10 nm do 380 nm) - promieniowanie ultrafioletowe jest zaliczane do promieniowania jonizującego, czyli ma zdolność odrywania elektronów od atomów i cząsteczek. W technice ultrafiolet stosowany jest powszechnie. Powoduje świecenie (fluorescencję) wielu substancji chemicznych. W świetlówkach ultrafiolet wytworzony na skutek wyładowania jarzeniowego pobudza luminofor do świecenia w zakresie widzialnym. Niektóre owady, na przykład pszczoły, widzą w bliskiej światłu widzialnemu części widma promieniowania ultrafioletowego, również rośliny posiadają receptory ultrafioletu. 
	\item \underline{Promieniowanie rentgenowskie} (od 5 pm do 10 nm) - promieniowanie rentgenowskie jest promieniowaniem jonizującym. Technicznie promieniowanie rentgenowskie uzyskuje się przeważnie poprzez wyhamowywanie rozpędzonych cząstek naładowanych. W lampach rentgenowskich są to rozpędzone za pomocą wysokiego napięcia elektrony hamowane na metalowych anodach. Źródłem wysokoenergetycznego promieniowania rentgenowskiego są również przyspieszane w akceleratorach cząstki naładowane. Promieniowanie rentgenowskie jest wykorzystywane do wykonywania zdjęć rentgenowskich do celów defektoskopii i diagnostyki medycznej. W zakresie promieniowania rentgenowskiego są również prowadzone obserwacje astronomiczne. 
	\item \underline{Promieniowanie gamma} (0,03 pm do 300 pm) - promieniowania gamma jest promieniowaniem jonizującym. Promieniowanie gamma towarzyszy reakcjom jądrowym, powstaje w wyniku anihilacji – zderzenie cząstki i antycząstki, oraz rozpadów cząstek elementarnych. Otrzymywane w cyklotronach promieniowanie hamowania i synchrotronowe również leży w zakresie długości fali promieniowania gamma, choć niekiedy bywa nazywane wysokoenergetycznym promieniowaniem rentgenowskim. Promienie gamma mogą służyć do sterylizacji żywności i sprzętu medycznego. W medycynie używa się ich w radioterapii oraz w diagnostyce. Zastosowanie w przemyśle obejmują badania defektoskopowe. Astronomia promieniowania gamma zajmuje się obserwacjami w tym zakresie długości fal. 
\end{enumerate}