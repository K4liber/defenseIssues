
\begin{enumerate}[1]
\item \underline{Zasada dynamiki Newtona:}

I zasada dynamiki Newtona zakłada istnienie inercjalnego układu odniesienia. Układ inercjalny to taki, w którym cząstka nie podlegająca oddziaływaniu z otoczeniem, spoczywa lub porusza się po prostej ze stałą predkością (układy inercjalne poruszają się ruchem jednostajnym lub spoczywają względem siebie).

\item \underline{Zasada dynamiki Newtona:}
	
W inercjalnym układzie odniesienia jeśli siły działające na ciało nie rownoważą się ($ \vec{F_w} \neq 0 $) to ciało porusza się z przyśpieszeniem wprost proporcjonalnym do siły wypadkowej, a odwrotnie proporcjonalnym do masy ciała:\newline
$ \vec{a} = \frac{1}{m}*\vec{F_w} $\newline
$ \vec{F_w} = \frac{d\vec{p}}{dt} = \frac{d}{dt}(m\vec{v}) = m\frac{d\vec{v}}{dt} = m\vec{a} $\newline
Pierwsza zasada dynamiki Newtona jest szczególnym przypadkiem drugiej zasady dynamiki Newtona (gdy $ \vec{F_w} = 0 $).

\item \underline{Zasada dynamiki Newtona:}

Oddziaływania ciał są zawsze wzajemne. Jeżeli ciało \textit{A} działa na ciało \textit{B} siła $\vec{F}$ (akcja), to ciało \textit{B} działa na ciało \textit{A} siłą o takiej samej wartości i kierunku, lecz przeciwnym zwrocie (reakcja).
	
\end{enumerate}

\underline{Szczególna teoria względności}

\begin{enumerate}[1]
	\item \underline{postulat}:
	
We wszystkich układach inercjalnych prawa fizyki są jednakowe (zasada względności).

	\item \underline{postulat}:
	
Dla wszystkich obserwatorów inercjalnych prędkość światła w próżni (\textit{c}) jest taka sama i nie zależy od prędkości źródła światła.

\end{enumerate}

Te postualaty Einsteina prowadzą do tranformacji Lorentza:\newline
Rozważmy układ K oraz układ K' poruszający się względem K z predkością $ v_x $ wzdłuż osi OX (dla t = t' = 0 początki układów współrzednych $ 0_K $ i $ 0_{K'} $ pokrywają się), wtedy:\newline
$ t' = \gamma(\textit{t} - \frac{v_xx}{c^2}) $, $\gamma = \frac{1}{\sqrt{1-\frac{v_x^2}{c^2}}}$\newline
$ x'=\gamma(x - v_xt) $, y'=y, z'=z

Konsekwencje szczególnej teorii względności:

\begin{enumerate}[-]
\item Względność jednoczesności - dwa zdarzenia określone przez jednego obserwatora jako jednoczesne, mogą nie być jednoczesne dla innego obserwatora.
\item Dylatacja czasu - czas, jaki mija pomiędzy dwoma zdarzeniami, nie jest jednoznacznie określony, lecz zależy od ruchu obserwatora (paradoks bliźniąt).
\item Relatywistyczne składanie prędkości.
\item Masa jest równoważna energii $ E=mc^2 $.
\item Ciała bezmasowe poruszają się z prędkością c, dla ciał z niezerową masą niemożliwe jest osiągnięcie prędkości c.
\item Skrócenie Lorentza.
\end{enumerate}