\underline{Zerowa zasada termodynamiki} - jeśli układy A i B mogące ze sobą wymieniać ciepło si ą ze sobą w równowadze termicznej, i to samo jest prawdą dla układów B i C, to układy A i C również są ze sobą w równowadze termicznej. Z zasady tej wynika istnienie temperatury empirycznej. Temperatura to taka wielkość fizyczna, która dla układów A, B i C jest równa, gdy ustaje przepływ ciepła. Układy będą ze sobą w równowadze termodynamicznej. Zerowa zasada termodynamiki stwierdza także, że ciało w równowadze termodynamicznej ma wszędzie tę samą temperaturę.

\underline{Pierwsza zasada termodynamiki} - zmiana energii wewnętrznej (będącej funkcją stanu) układu zamkniętego (nie wymienia masy z otoczeniem) jest równa energii, która przepływa przez jego granice na sposób ciepła i pracy:\newline
$ \Delta U = W + Q $, gdzie:\newline
$ \Delta U $ - zmiana energii wewnętrznej układu,\newline
$ W $ - praca wykonana na układzie,\newline
$ Q $ - ciepło przekazane do układu.\newline
Alternatywne sformułowanie - nie istnieje perpetuum mobile pierwszego rodzaju (maszyna, która wytwarza więcej energii, niż sama zużywa).

\underline{Druga zasada termodynamiki} - w układzie termodynamicznie izolowanym istnieje funkcja stanu zwana entropią, która nie maleje z czasem. Matematyczny zapis tego faktu to następujące sformułowanie: zmiana entropii $ \Delta S $ w dowolnym procesie odwracalnym jest równa całce z przekazu ciepła DQ podzielonego przez temperaturę T. W procesie nieodwracalnym natomiast zmiana entropii jest większa od tej całki:
$ \Delta S \geq \int \frac{DQ}{T} $\newline
Różnica ta jest miarą nieodwracalności procesu i jest związana z rozpraszaniem energii. Entropia (S) jest funkcją stanu będąca miarą liczby sposobów (W), na jakie może być zrealizowany określony stan termodynamiczny danego układu w określonej temperaturze (T). Układ dąży do stanu, który może być w danych warunkach zrealizowany na jak najwięcej sposobów; dąży więc on do maksymalizacji entropii. Nie można bez wkładu pracy przesyłać energii termicznej między ciałami mającymi tę samą temperaturę. Oznacza to, że perpetuum mobile drugiego rodzaju (maszyna, która zamienia energię cieplną na pracę mechaniczną bez wzrostu całkowitej entropii) nie istnieje. 

\underline{Trzecia zasada termodynamiki} - nie można za pomocą skończonej liczby kroków uzyskać temperatury zera bezwzględnego (zero kelwinów), jeżeli za punkt wyjścia obierzemy niezerową temperaturę bezwzględną. Podstawą takiego zdefiniowania III zasady termodynamiki jest analiza sprawności lodówki. Jak wiemy, lodówka działa na zasadzie odwrotnego cyklu Carnota, a jej sprawność dana jest wzorem:\newline
$ n = \frac{Q_{odebrane}}{W} = \frac{T_2}{T_1-T_2} $\newline
Jeżeli ciało o określonej temperaturze $ T_1 $ chcielibyśmy schłodzić do $ T_2 \to $, odbierając przy tym skończone ciepło $ Q $, to analizując wzór widzimy, że w takim wypadku $ \frac{Q}{W} \to 0 $, czyli $ W\to $ nieskończoności. Gdybyśmy podstawili $ T_2 = 0 $, równanie nie miałoby sensu matematycznego, co oznacza, że nie da się osiągnąć temperatury zera bezwzględnego w skończonej liczbie kroków. Mówiąc inaczej, gdyby udało się schłodzić jakąś substancję do 0 K i gdyby utworzyła ona kryształ doskonały nieposiadający zamrożonych defektów krystalicznych, to jej entropia musiałaby przyjąć wartość 0. Jest to jednak technicznie, a także formalnie, niewykonalne:
$ \lim_{T\to0} S = 0 $