\underline{Dualizm korpuskularno-falowy} – cecha obiektów kwantowych (np. fotonów czy elektronów) polegająca na przejawianiu, w zależności od sytuacji, właściwości falowych (dyfrakcja, interferencja) lub korpuskularnych (dobrze określona lokalizacja, pęd).

Zgodnie z mechaniką kwantową cała materia charakteryzuje się takim dualizmem, chociaż uwidacznia się on bezpośrednio tylko w bardzo subtelnych eksperymentach wykonywanych na atomach, fotonach, czy innych obiektach kwantowych.

Dualizm korpuskularno-falowy jest ściśle związany z falami de Broglie’a – koncepcją, która przyczyniła się do powstania mechaniki kwantowej, a w szczególności do wyprowadzenia równania Schrödingera: $ \lambda = \frac{h}{p} $, gdzie gdzie $ h $ jest stałą Plancka, łączącą wielkości falowe (długość fali $ \lambda $) z korpuskularnymi (pęd $ p $).

Dokonując pomiaru położenia cząstki zawsze znajdujemy ją w przybliżeniu w konkretnym miejscu w przestrzeni (rejestruje ją konkretny detektor). W przypadku \underline{eksperymentów z podwójną szczeliną} uzyskuje się interferencję bądź nie w zależności od tego czy obiekt przejawia właściwości falowe czy cząsteczkowe. Właściwości cząsteczkowe są obserwowane, gdy w szczelinach będzie umieszczony detektor, wykrywający przez którą szczelinę się poruszał obiekt. Przyczyną tego jest istnienie splątania kwantowego i dostępność informacji o obserwablach. Po detekcji cząstki nieoznaczoność jej pędu stopniowo wzrasta, przez co maleje widoczność prążków interferencyjnych.

Young w 1801 r. pierwszy eksperyment przeprowadził ze światłem. Następnie przeprowadzano eksperymenty z elektronami, atomami oraz cząsteczkami. Największe układy, dla których zaobserwowano dualizm korpuskularno-falowy miały 58 (ftalocyjanina), 114 atomów (pochodna ftalocjaniny) i nawet po kilkaset atomów.

